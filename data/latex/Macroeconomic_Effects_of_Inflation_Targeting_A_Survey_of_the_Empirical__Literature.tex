\documentclass{article}
\usepackage{amsmath}
\usepackage{graphicx}
\usepackage{natbib}

\title{Macroeconomic Effects of Inflation Targeting: A Survey of the Empirical Literature}
\author{Goran Petrevski \\
        Ss. Cyril and Methodius University in Skopje, Faculty of Economics \\
        Orchid ID: 0000-0002-5583-8287 \\
        E-mail: goran@eccf.ukim.edu.mk}

\begin{document}
\maketitle

\begin{abstract}
This paper surveys the voluminous empirical literature on inflation targeting (IT). Specifically, the paper focuses on three main issues: the main institutional, macroeconomic, and technical determinants that affect the adoption of IT; the effects of IT on macroeconomic performance (inflation expectations, inflation persistence, average inflation rate, inflation variability, output growth, output volatility, interest rates, exchange rates, and fiscal outcomes); and disinflation costs of IT (the so-called sacrifice ratios). The main findings from our review are the following: concerning the determinants behind the adoption of IT, there is robust empirical evidence that larger and more developed countries are more likely to adopt the IT regime; similarly, the introduction of this regime is conditional on previous disinflation, greater exchange rate flexibility, central bank independence, and a higher level of financial development; however, the literature suggests that the link between various macroeconomic and institutional determinants and the likelihood of adopting IT may be rather weak, i.e., they are not to be viewed either as strict necessary or sufficient conditions; the empirical evidence has failed to provide convincing evidence that IT itself may serve as an effective tool for stabilizing inflation expectations and for reducing inflation persistence; the empirical research focused on advanced economies has failed to provide convincing evidence on the beneficial effects of IT on inflation performance, concluding that inflation targeters only converged towards the monetary policy of non-targeters, while there is some evidence that the gains from the IT regime may have been more prevalent in the emerging market economies (EMEs); there is not convincing evidence that IT is associated with either higher output growth or lower output variability; the empirical research suggests that IT may have differential effects on exchange-rate volatility in advanced economies versus EMEs; although the empirical evidence on the impact of IT on fiscal policy is quite limited, it supports the idea that IT indeed improves fiscal discipline; the empirical support to the proposition that IT is associated with lower disinflation costs seems to be rather weak. Therefore, the accumulated empirical literature implies that IT does not produce superior macroeconomic benefits in comparison with the alternative monetary strategies or, at most, they are quite modest.
\end{abstract}

\section{Introduction}

In 1990, New Zealand adopted IT as a new framework for conducting monetary policy, followed by Canada (1991), United Kingdom (1992), and several other industrialized countries. Since then, this monetary policy regime has gained increasing popularity in both advanced countries and EMEs. As a result, by 2019, in one form or another, IT has been implemented in 43 countries as diverse as Albania, Ghana, Mexico, Russia, South Africa, Sweden, and UK \citep{IMF2020}. Commenting on the global popularity of IT, \citet{rose2007} claims that the international monetary system has been dominated by inflation targeters. Also, IT is the longest-lasting monetary strategy after World War II. In addition, unlike the other monetary strategies, IT has proved to be durable as no country has left it, yet. Therefore, according to \citet{walsh2009}, the actual experience with IT unambiguously shows that it is both feasible and sustainable.

Table 1 in the Appendix presents the IT adoption dates for selected industrialized countries and EMEs. As can be seen, in several cases it is quite difficult to specify the exact date of adopting the IT regime, mainly due to the variations in its practical implementation. For instance, Chile introduced IT in 1990-1991, while retaining its exchange rate band by August 1999, when it switched to full-fledged IT; similarly, Israel adopted an explicit inflation target in 1992, but retained the exchange rate band through 1997; Mexico, too, introduced some elements of IT in 1999 though it had not moved to the full-fledged variant until 2002. In these regards, not only the adoption dates differ among the individual empirical studies, but there are also discrepancies between the dates that can be found in the empirical literature and those specified on the central bank’s websites and in the official documents.

According to \citet{bernanke1997}, \citet{hammond2012}, \citet{mishkin2000}, \citet{mishkin1997}, and \citet{svensson2002, svensson2010}, practical implementation of IT is characterized by the following common features: strong institutional commitment to price stability as the primary monetary policy objective in the medium-to-long run; the announcements of explicit numerical inflation targets for medium-term inflation; short-term flexibility, which allows the policymakers to respond to short-term disturbances from various sources (supply shocks, exchange rate changes, etc.); as well as a high degree of central bank independence, accountability, and transparency. As for the practical implementation of monetary policy, a distinctive feature of IT is the absence of intermediate targets, which stands in sharp contrast to the alternative monetary policy strategies, such as monetary or exchange rate targeting. In these regards, \citet{bernanke1997} describes IT as a “rule-like strategy” or “constrained discretion”, which enables the central bank to be focused on price stability while at the same time being able to deal with short-run macroeconomic fluctuations.

Consequently, IT is supposed to provide the following advantages in comparison with other monetary policy regimes: it builds discipline, credibility, and accountability of central banks by preventing policymakers from engaging in systematic short-term stimuli, and by subjecting the central bank’s short-run actions to public scrutiny and debate about their long-term consequences; it improves the central bank’s communication with the general public; it is both an efficient and forward-looking strategy as it uses all the available information along with an explicit account of time-lags; it helps the central bank to anchor inflation expectations and to cope with adverse supply shocks, which results in lower economic costs \citep{Batini2007, Bernanke1997, Mishkin2000}. Similarly, \citet{mishkin2002} argue that IT offers several benefits for EMEs, such as reinforcing central bank independence and enabling central banks to be more focused on inflation. In addition, \citet{thornton2017} show that IT facilitates the implementation of countercyclical monetary policy in these countries, the majority of which have been previously notorious for implementing procyclical policies.

The proponents of IT often emphasize its flexibility as a crucial property in the practical implementation of monetary policy \citep{mishkin1999}. On the one hand, the firm focus on price stability increases the credibility of central banks with its favourable effects on inflation expectations. On the other hand, central banks typically approach the inflation target gradually over time, thus, retaining the manoeuvre room for responding to possible adverse short-run circumstances. In other words, within this policy framework, central banks can combine the inflation targets with other policy goals such as output or employment \citep{agenor2002, Leiderman1995, Svensson1997a}. In this way, flexible IT appears to be an optimal monetary policy leading to lower average inflation accompanied by output stabilization \citep{Ball1999a, Ball1999b, Clarida1999, Svensson1997b}.

On the other hand, IT has been criticized on various grounds: a) it is too rigid by constraining discretion in monetary policy, thus, unnecessarily restraining growth; b) it cannot anchor inflation expectations because it offers too much discretion with respect to both the definition and maintenance of inflation targets; c) relatively frequent misses of inflation targets, due to the imperfect control of inflation and the long lags in the monetary transmission mechanism, can lead to weak central bank credibility; d) it may not be sufficient to ensure fiscal discipline or prevent fiscal dominance; e) the exchange rate flexibility required by IT might cause financial instability; and f) its practical implementation is dependent on a number of institutional and technical preconditions, which are not met in most of EMEs \citep{Batini2007, Bernanke1999, Mishkin1999, Mishkin2000}. The last four disadvantages are especially relevant for EMEs. IT has been implemented in EMEs within a specific macroeconomic and institutional environment, which undoubtedly affects the implementation of effective monetary policy. For instance, most of these countries are characterized by fiscal dominance and weak banking systems, which are not consistent with a sustainable IT regime. Also, as the long history of high inflation undermines the central banks’ credibility, the introduction of IT should be preceded by, at least partial, disinflation. Finally, simultaneously with inflation targets, central banks in EMEs should take care of smoothing excessive exchange rate fluctuations for at least two reasons: the exchange rate channel is of crucial importance in small open economies; both dollarization and the exposure of sudden stops of foreign capital amplifies the vulnerability of firms and banks to exchange rate fluctuations, which may lead to a full-blown financial crisis. Hence, central banks in EMEs must be concerned with exchange rate fluctuations, thus preventing sharp depreciations that might cause high inflation and financial instability. Yet, putting too much emphasis on the exchange rates might create confusion in the public, thus, compromising the credibility of inflation targets. Therefore, the practical implementation of IT in EMEs can be quite complicated: on the one hand, central banks should smooth exchange rate fluctuations, but on the other hand, they cannot allow the inflation targets to be subordinated to exchange rate policy \citep{Agenor2002, Buffie2018, Aizenman2010, Carrare2002, Cespedes2014, Civcir2010, Edwards2007, Eichengreen1999, Ho2003, Kumhof2007, Leiderman2006, Jonas2007, Mishkin2000, Mishkin2004b, Mishkin2002, Parrado2004, Roger2009, Schaechter2000, Schmidt-Hebbel2002, Siregar2010, Stone2009}.

Given these unfavourable macroeconomic and institutional conditions prevailing in EMEs, IT need not necessarily provide the outcomes that are either a priori expected in theory or observed in advanced economies. At the same time, as suggested by \citet{walsh2009}, the larger variation in inflation experiences in EMEs may help identify the true effects of IT. Consequently, a large body of empirical evidence on the macroeconomic effects of IT has focused on EMEs.

\section{Main determinants of the choice of IT}

Although an increasing number of countries have adopted IT during the past two decades, many more of them, especially the EMEs, still rely on other strategies for controlling inflation. This naturally raises the question of which factors determine the choice of IT vis-a-vis alternative monetary policy regimes. In principle, this choice should be based on both theoretic grounds and empirical evidence. Theoretically, the choice of optimal monetary policy has been analyzed within a well-specified (usually, a small-scale) macroeconomic model by comparing the central bank’s loss function under alternative policy rules. Here, a number of papers demonstrate that IT outperforms the alternative monetary policy rules in terms of inflation/output variability \citep{Ball1999a, Ball1999b, Haldane1999, Rudebusch1999, Svensson1999a, Svensson1999b, Svensson2000}. At the same time, despite the accumulated empirical evidence on the macroeconomic effects of IT, the findings from these studies are rather inconclusive: while some papers suggest that IT is associated with both lower average inflation and improved inflation/output variability, others show that it does not produce superior macroeconomic benefits or, at most, they are quite modest (See the empirical evidence reviewed in Section 3). Therefore, the increasing adoption of IT is not based on strong empirical evidence with respect to the macroeconomic performance of this monetary regime. In addition, it should be noted that the experience of advanced countries may not be relevant for EMEs due to their specific institutional and macroeconomic characteristics.

The early literature has suggested that the adoption of IT requires the fulfilment of several economic, institutional, and technical prerequisites, such as: the absence of fiscal dominance, strong external position, relatively low inflation, well-developed financial markets and sound financial system, central bank independence, some structural characteristics (price deregulation, low dollarization, low sensitivity to supply shocks, strong external position etc.), the absence of de facto exchange rate targets, well developed technical infrastructure for forecasting inflation etc. \citep{agenor2002, amato2002, carare2002, carare2006, eichengreen1999, freedman2009, Freedman2010, IMF2006, Masson1997, Mishkin2000, Mishkin2002, Mishkin2007}.

While theoretically sound, the experience shows that many inflation targeters, especially the EMEs, have not met all these requirements, at least in the initial phase. Indeed, EMEs operate in a specific institutional and macroeconomic environment, which often complicates the design and implementation of the IT regime. For instance, the presence of fiscal dominance, a common feature in many EMEs, undermines the effectiveness of monetary policy. Similarly, weak banking systems in these countries often preclude the use of market-based monetary policy instruments. Further on, the long historical experience with high inflation reduces the credibility of their central banks, requiring at least partial disinflation before the introduction of IT. Also, given the crucial importance of the exchange rate channel in small open economies, the central banks in EMEs must be concerned with both exchange rate fluctuations and inflation targets simultaneously. Finally, central banks in many EMEs often lack the necessary technical infrastructure (data availability, lack of systematic forecasting process, low understanding of the transmission mechanism etc.), which hampers the day-to-day implementation of IT \citep{Amato2002, Jonas2007, Masson1997, Mishkin2000, Mishkin2004, Mishkin2002}.

In this regard, \citet{masson1997} assess the monetary policy framework in five EMEs and show that they lag substantially by those prevailing among the inflation targeters. Therefore, they conclude that developing countries do not fulfil the requirements for adopting IT. Similarly, based on a survey of 31 central banks, \citet{batini2007} assess whether some preconditions must be met before adopting IT in EMEs, such as: technical infrastructure, financial system, institutional central bank independence, and economic structure. They construct an extensive list of parameters and, by quantifying each of them, conclude that EMEs had not satisfied these required preconditions, which implies that adopting IT does not depend on meeting some strict initial pre-conditions. Similarly, based on the experience with the introduction and implementation of IT, \citet{freedman2009} and \citet{schmidt-hebbel2016} argue that a country must meet some basic preconditions before adopting IT, though most of the countries failed to meet all the preconditions. More importantly, they show that the adoption of IT itself promotes the fulfilment of these preconditions. \citet{samarina2014} provide a strong empirical support to this hypothesis by showing that there is a structural change after the adoption of IT, implying that, even when a country does not meet all the preconditions, once it has adopted IT, this decision leads to changes in the institutions which support its proper functioning.

The empirical literature on the determinants behind the adoption of IT generally follows an eclectic approach by specifying a general list of determinants that are expected to affect the choice of IT. In other words, only a few studies focus on the role of specific factors (e.g., political). Consequently, this approach prevents us from providing a structured survey of this strand of literature. In addition to the empirical studies investigating explicitly the determinants of IT, there are several papers which, although having different research topics, deal with issues as part of the overall empirical approach. Here, we refer to the papers employing the propensity score matching methodology or other similar types of treatment effects regression. Within this framework, in the first stage of the empirical investigation the dummy variable of adopting IT is usually regressed on several macroeconomic variables. The non-exhaustive list of this research includes \citet{ardakani2018}, \citet{arsic2022}, \citet{demendonca2012}, \citet{fry-mckibbin2014}, \citet{goncalves2009}, \citet{lin2010}, \citet{lin2007, lin2009}, \citet{lucotte2012}, \citet{minea2014}, \citet{minea2021}, \citet{mukherjee2008}, \citet{pontines2013}, \citet{samarina2014}, \citet{vega2005}, and \citet{yamada2013}.

In what follows we first provide a brief explanation of the expected impact of the above-mentioned determinants on the likelihood of adopting IT along with an overview of the main findings from the empirical research in this field. Table 2A in the Appendix provides detailed descriptions of individual studies, while Table 2B summarizes the main empirical findings by each determinant.

Besides the well-known argument that small open economies are the most serious candidates for pegged exchange rates, a priori, it is difficult to say whether IT is a “one-size-fits-all” strategy which is appropriate for both large and small economies. At the same time, the experience reveals that IT has been implemented in a wide array of countries, ranging from very small (Albania, Israel, Serbia etc.) to very large countries (Brazil, India, Russia, and South Africa). Most of the empirical studies confirm that size matters for the adoption of IT by showing that the size of the economy, measured either by the level of GDP or GDP per capita, is associated with a higher likelihood of adopting this monetary regime \citep{deMendonca2012, Leyva2008, Lucotte2010, Minea2021, Samarina2014, Yamada2013}. Yet, since GDP per capita is used as a general proxy for the level of economic development, some of these results imply that not larger but more developed economies are more likely to adopt IT. In a similar fashion, several studies obtain the same findings working with either area or population size \citep{Arsic2022, Rose2014, Wang2016, Yamada2013}. However, this conclusion is not shared by \citet{hu2006} and \citet{ismailov2016}, while some papers suggest that the importance of these determinants may be sensitive to the sample. For instance, \citet{carare2006} and \citet{fouejieu2017} are not able to confirm this hypothesis for the sample of EMEs, while \citet{samarina2014} provide similar evidence for advanced economies.

Usually, central banks tend to choose their monetary strategy in response to past macroeconomic performance. In theory, inflation and output are the standard elements in the central bank’s loss function. In practice, although inflation control is the primary goal of monetary policy, central banks often pay attention to economic activity, too. In this regard, if a country has experienced unsatisfactory economic performance, such as low growth rates or high output volatility, then the central bank might consider switching to IT as a strategy which enables policymakers to focus on the developments in the real economy, too. This argument may be especially relevant for EMEs, which traditionally have worse performance than advanced economies due to the unfavourable macroeconomic environment prevailing in them \citep{Fraga2003}. The empirical literature offers mixed evidence on the effects of economic performance on the choice of IT. For instance, \citet{lucotte2012} finds that higher output growth increases the likelihood of adopting IT. On the other hand, a few papers find a negative association between GDP growth and the probability of adopting IT, implying that the countries experiencing satisfactory economic performance have fewer incentives to switch to this regime \citep{Ardakani2018, Hu2006}. In fact, most of the empirical research has produced either statistically insignificant or non-robust results about the importance of output growth for the choice of IT \citep{Fouejieu2017, Lin2010, Lin2007, Lin2009, Pontines2013, Samarina2014, Thornton2017, Wang2016}. Similarly, the empirical literature does not provide an unambiguous answer to the question of whether the countries facing more (less) stable economic conditions (measured by output volatility) are good (bad) candidates to implement IT. \citet{mukherjee2008} as well as \citet{samarina2014} show that higher output volatility makes adopting IT more likely, \citet{hu2006} finds that this factor is statistically insignificant, while \citet{samarina2014} and \citet{stojanovikj2019} obtain opposite findings, finding that macroeconomic instability reduces the likelihood of adopting IT.

Similarly, the central bank might choose its monetary policy strategy based on its experience with past inflation rate. Here, the literature suggests that the introduction of IT is not feasible at high inflation rates, when there is a considerable degree of inertia in nominal variables, and monetary policy is largely accommodative. Therefore, a country should first reduce inflation to a relatively low level before it adopts this monetary regime \citep{Carare2002, Masson1997, Mishkin2000}. Accordingly, this argument implies that higher inflation rates make the introduction of IT less likely. The empirical research provides strong support to this proposition for both industrialized countries and EMEs \citep{Ardakani2018, Arsic2022, Hu2006, Lin2010, Lin2007, Lin2009, Minea2014, Minea2021, Pontines2013, Samarina2014, Thornton2017}. Yet, it is fair to note that the empirical evidence is not unanimous: \citet{goncalves2009} and \citet{vega2005} obtain opposite findings, while the results in \citet{fry-mckibbin2014}, \citet{samarina2014}, \citet{samarina2014}, and \citet{wang2016} are either insignificant or sensitive to the sample they work with (industrialized countries versus EMEs). Given the consensus view of inflation as a monetary phenomenon in the long-run, several studies test the relationship between money growth and the probability of adopting IT. It seems that this consensus prevails in the empirical literature, too \citep{Ardakani2018, Arsic2022, Lin2010, Lin2007, Lin2009, Pontines2013, Samarina2014, Yamada2013}, with only a few exceptions \citep{Fry-McKibbin2014, Wang2016}. Therefore, it is safe to say that the countries experiencing higher past inflation or, equivalently, higher money growth, are less likely to switch to IT.

A strong external position, too, is expected to make the adoption of IT more likely. Within this monetary regime, the central bank should be focused on achieving and maintaining the inflation targets, which is only possible if the concerns for the balance payment and the exchange rate are subordinated to the primary objective of monetary policy \citep{Carare2002}. However, the available empirical literature provides ambiguous findings on the importance of external macroeconomic conditions for the adoption of IT. For instance, \citet{arsic2022} show that a strong current account position reduces the likelihood of adopting IT, \citet{mukherjee2008} obtain opposite findings, while this variable is not significant in \citet{hu2006}. The evidence on the role of external debt is equally inconclusive: \citet{hu2006} finds that higher external debt reduces the probability of adopting IT, while this factor is not significant in \citet{samarina2014}. Working with a sample of EMEs, \citet{yamada2013} finds that foreign exchange reserve, too, is not a significant factor when switching to IT. Similarly, \citet{ardakani2018} obtain opposite results on the importance of central bank’s assets: while their size makes the adoption of IT more likely in the advanced countries, it is quite contrary for the case of developing countries. In fact, all this evidence suggests that, although a strong external position may make the transition toward IT easier, the developing countries characterized by a favorable current account balance and/or sizable foreign exchange reserves have fewer incentives to change their existing monetary regimes (usually, some variant of a currency peg).

In addition, fiscal discipline is often listed as one of the basic requirements for adopting IT. In the presence of persistent high fiscal deficits, the central bank may pursue accommodative monetary policy, which clearly undermines its ability to meet the announced inflation targets. Similarly, a high level of public debt may provide an incentive for the government to reduce the real value of the debt by high inflation \citep{Mishkin2000}. As a result, fiscal discipline, and sound public finance in general (efficient tax-collection procedures, high government revenue, low budget deficits, and low public debt), are expected to increase the likelihood of adopting IT. However, the empirical evidence on the role of fiscal discipline is rather mixed, and this is equally true for the importance of both the budget balance and the public debt. As for the role of the budget balance, only a few studies support the above proposition \citep{Hu2006, Lin2010}, while the majority of the empirical research either rejects it \citep{deMendonca2012, Pontines2013} or provides inconclusive evidence \citep{Carare2006, Leyva2008, Lin2007, Mishkin2002, Samarina2014, Vega2005}. The empirical literature is equally inconclusive when employing government debt as a fiscal policy indicator. Here, only a few studies show that higher indebtedness reduces the probability of adopting IT \citep{Goncalves2009, Minea2014, Thornton2017}, while others obtain either opposite findings \citep{Arsic2022} or provide mixed evidence \citep{Carare2006, Ismailov2016, Lin2009, Samarina2014, Samarina2014, Wang2016}. Therefore, the empirical evidence implies that the role of fiscal discipline in choosing IT may be conditional on other factors, such as a country’s history with inflation, the government’s access to financial markets, central bank independence, the limits on central bank lending to the government etc. For instance, in countries with a long history of low inflation and with broad markets for government debt, the credibility of IT is less dependent on the government’s actual fiscal position. Also, central bank independence accompanied by clear limits on central bank lending to the government diminish the role of fiscal discipline in the decision-process \citep{Carare2002}. Therefore, the inconclusive evidence on the role of fiscal discipline for adopting IT may not be surprising for the case of industrialized countries, which are characterized by a long history of low inflation, broad and deep markets for government debt, and a strong institutional environment. However, the lack of firm evidence is puzzling for the case of developing countries despite their long record of fiscal dominance, high inflation, and low central bank credibility.

Both trade and financial openness of the economy are also considered relevant factors for the choice of monetary policy strategy. For instance, many EMEs are traditionally exposed to large and persistent exogenous shocks, which makes them very sensitive to commodity prices and exchange rate fluctuations \citep{Fraga2003}. Consequently, small open economies tend to choose currency pegs as a preferred monetary regime, thus, being less likely to switch to IT \citep{IMF2006, Rose2014}. As for the importance of trade openness, the empirical literature seems to be completely divided about the importance of this factor: while a few studies find that trade openness is associated with a higher probability of adopting IT \citep{Leyva2008, Lucotte2010, Mishkin2002}, the majority of empirical research fails to support this proposition \citep{Arsic2022, deMendonca2012, Fouejieu2017, Fry-McKibbin2014, Hu2006, Lucotte2012, Lin2010, Lin2007, Lin2009, Minea2021, Minea2014, Rose2014, Samarina2014, Samarina2014, Thornton2017, Vega2005}, and this is true for both advanced economies and EMEs. On the other hand, most of the empirical suggests that financial openness is an important precondition for adopting IT \citep{deMendonca2012, Samarina2014, Thornton2017} though a few studies refute this conclusion \citep{Rose2014, Samarina2014, Samarina2014}.

Within the IT framework, price stability is the primary objective of monetary policy with other objectives (employment, exchange rate, external position) being subordinated to the inflation target. Therefore, by definition, IT requires flexible exchange rates, i.e., it is inconsistent with fixed exchange rate regimes. In other words, the presence of fixed exchange rates is expected to decrease the likelihood of adopting IT, while greater exchange rate flexibility works in the opposite direction. Indeed, the available empirical literature unanimously confirms that fixed exchange rates are not conducive to IT \citep{Ardakani2018, Arsic2022, deMendonca2012, Fouejieu2017, Fry-McKibbin2014, Lin2010, Lin2007, Lin2009, Minea2014}. Similarly, with a few exceptions \citep{Hu2006, Samarina2014}, the large majority of empirical research finds that exchange rate flexibility makes the adoption of IT more likely \citep{Ismailov2016, Lucotte2010, Lucotte2012, Minea2021, Mukherjee2008, Pontines2013, Samarina2014, Samarina2014, Thornton2017, Vega2005}. Therefore, we can conclude that there is a strong consensus that IT requires a higher degree of exchange rate flexibility, i.e., currency pegs are not compatible with this monetary regime.

Undoubtedly, central bank independence appears to be one of the most important institutional factors necessary for the successful implementation of IT. It is also understood that the central bank should have a clear mandate to pursue price stability with all the other objectives being subordinated to the inflation target \citep{Agenor2002}. It seems that this proposition has found widespread empirical support \citep{Fouejieu2017, Lin2007, Lucotte2010, Lucotte2012, Minea2014} with only a few dissenting studies \citep{Hu2006, Lin2009}. Here, the literature refers to the so-called instrument independence, i.e., the autonomy of the central bank in choosing its instruments to achieve the inflation targets. In one of the first attempts to address the factors behind the choice of IT, \citet{mishkin2002}, who conduct a cross-section analysis on a sample of 27 advanced countries and EMEs during the 1990s. The main findings from their study indicate that it is the type of central bank independence that matters for the choice of IT. Specifically, they find that legal central bank independence is not significant in the choice of IT. In addition, they show that the likelihood of adopting IT is positively associated with instrument independence, but goal independence has the opposite impact. \citet{samarina2014}, too, confirm the importance of instrument independence for the sample of developing countries, but they find that this type of central bank independence is not significant for advanced economies. \citet{carare2006} investigate one particular dimension of central bank independence – restrictions on government lending and find that it is important only for the EMEs. Finally, \citet{mukherjee2008} show that central banks with a clear focus on price stability, i.e., those without bank regulatory authority, are more likely to choose IT.

Financial development and financial stability facilitate the adoption of IT in many ways. A well-developed financial system not only enables the central bank to employ market-based instruments, but it has a central role in the monetary policy transmission mechanism. Also, operating in a sound financial system, the central bank is free from the responsibility to inject liquidity to the failing financial institutions, so that it can focus on the achievement of the announced inflation target \citep{Battini2007, Carare2002}. The available empirical evidence generally supports the proposition that a higher level of financial development is required for introducing IT \citep{Carare2006, deMendonca2012, Leyva2008, Samarina2014, Samarina2014, Thornton2017, Vega2005} although it is fair to say that the empirical support is far from unanimous \citep{Ardakani2018, Hu2006, Lucotte2010, Lucotte2012, Samarina2014}. On the other hand, \citet{samarina2014} find that financial structure (market-based versus bank-based financial systems) does not matter for adopting IT. Also, the empirical literature fails to provide a clear conclusion on the importance of financial (in)stability: \citet{samarina2014} find that the financial crisis dummy is statistically insignificant, while \citet{thornton2017} provide some weak evidence that this factor matters in developing countries only.

Further on, several studies focus on the importance of political institutions for the adoption of IT. For instance, comparing IT with exchange rate pegs for a large set of more than 170 countries, \citet{rose2014} show that IT is a preferred monetary regime for the countries with more developed democratic institutions. \citet{mukherjee2008} show that countries are more likely to adopt IT when the government and the central bank share the same preferences for tight monetary policy. Specifically, the combination of a right-leaning government and a central bank without bank regulatory authority is likely to be associated with the adoption of IT. \citet{ismailov2016} find that political stability does not affect the choice of IT for both low-income and high-income countries. \citet{lucotte2010, lucotte2012} investigates the role of institutional and political factors in adopting IT for a sample of 30 EMEs. His findings imply that several political determinants increase the likelihood of adopting IT, such as the number of veto players in the political system, political stability as well as federalism (decentralization). In a similar fashion, working with a sample of 53 developing countries, \citet{minea2021} obtain some other interesting results: on the one hand, they find that better institutional quality reduces the likelihood to switch to IT, while on the other hand, constraints on the executive make the introduction of IT more likely. In this respect, the former finding seems to be at odds with their theoretical model linking the monetary regime, quality of institutions, and the sources of government finance, while the latter result conforms well to the predictions from the theoretical model.

Finally, there are some technical prerequisites for the successful implementation of IT. For instance, the central bank should have a clear understanding of the time lag and the transmission mechanism; it should have long and reliable database and technical expertise to forecast inflation; it should conduct regular surveys of inflation expectations; and it should be able to develop market-based and forward-looking operating procedures. However, it is suggested that the initial technical conditions, although important, are not critical for introducing IT, i.e., the lack of these conditions can be remedied after the introduction of this monetary regime \citep{Battini2007, Carare2002, IMF2006, Mishkin2007}.

Unsurprisingly, the empirical research has not led to firm conclusions on the importance of each individual determinant for the adoption of IT reflecting the fact that developed countries and EMEs represent a heterogeneous group with different institutional and macroeconomic characteristics. Indeed, several studies show that the determinants of the choice of IT generally differ between advanced economies and EMEs. For instance, \citet{ardakani2018}, \citet{fry-mckibbin2014}, \citet{ismailov2016}, \citet{samarina2014}, \citet{samarina2014}, and \citet{thornton2017} find that some macroeconomic variables are relevant in both the advanced and developing countries, whereas others may have differential impacts across these two groups of countries.

As suggested above, the importance of various determinants of IT may differ with respect to the type of IT regimes. For instance, \citet{carare2006} review the global experience with IT by focusing on the factors affecting the evolution between various variants of this regime (“lite”, eclectic, and full-fledged). They find that the level of economic and financial development are the most significant factors for the overall central bank credibility and, thus, for the choice of IT regimes. Also, they discuss the experience of EMEs and show that the likelihood to move from “lite” to full-fledged IT is predominantly influenced by the level of financial development, government debt, and central bank restrictions on government financing. In their comprehensive study, \citet{samarina2014}, too, show that the most important factors behind the adoption of IT differ between soft and full-fledged inflation targeters. Specifically, they find that flexible exchange rate regimes, exchange rate volatility, central bank independence, and external debt affect the probability of adopting soft IT, whereas inflation, output growth, and public debt are the most important factors for adopting full-fledged IT.

Based on Table 2B in the Appendix, the main findings from the empirical literature can be summarized as follows: the empirical research generally suggests that larger economies are more likely to choose this monetary regime; also, the level of economic development is associated with a higher likelihood of adopting IT; these findings imply that IT may not be a feasible monetary regime for small and/or low-income countries; however, the empirical literature offers diverse results on the effects of economic activity, i.e., it is not clear whether the adoption of IT is more likely in the countries with higher or lower output growth; similarly, the empirical literature does not provide an unambiguous answer to the question of whether the countries facing more (less) stable economic conditions (measured by output volatility) are good (bad) candidates to introduce IT; on the other hand, there is strong empirical evidence that the countries experiencing higher past inflation or equivalently, higher money growth are less likely to switch to IT; this finding is consistent with the common requirement that the introduction of IT is conditional on previous disinflation; as for the external macroeconomic conditions, there are ambiguous findings on the importance of current account balance for the adoption of IT; further on, higher interest rates seem to increase the likelihood of introducing inflation rate though this finding need not be true for the long-term interest rates.

Concerning the exchange rate regime, there is a strong consensus that this monetary policy framework requires a higher degree of exchange rate flexibility, i.e., currency pegs are not conducive to IT; in addition, there is a consensus that a higher degree of central bank instrument-independence is a necessary condition for adopting IT; similarly, the empirical evidence generally supports the proposition that a higher level of financial development is required for introducing IT though this finding does not receive uniform empirical support; finally, the empirical research on the importance of political institutions suggests that democracy, decentralization, and political polarization all increase the likelihood of adopting IT; on the other hand, the literature seems to be completely divided about the importance of trade and financial openness – some studies find that IT is more likely in more open economies, while others reach the opposite conclusion; similarly, the empirical evidence on the role of fiscal discipline is rather mixed, and this is equally true for the importance of both the budget balance and the public debt.

The lack of robust findings in this field is not surprising at all. In fact, both the early literature and the experience of inflation targeters suggest that the link between various macroeconomic and institutional determinants and the probability of adopting IT might be weak, i.e., they are not to be viewed either as strict necessary or sufficient conditions. For instance, many EMEs introduced the IT starting from moderate inflation rates, ranging from 10\% to 40\% \citep{Mishkin2002}. Similarly, Brazil introduced the IT after the sharp devaluation in 1999, followed by fiscal and political instability \citep{Mishkin2004b, Mishkin2002}. In the late 1990s, Poland, Hungary, and the Czech Republic adopted the IT notwithstanding the large fiscal deficits. In addition, during the initial phase, Poland and Hungary implemented the IT in the presence of exchange-rate bands and with a limited capacity for forecasting inflation \citep{Jonas2007}. Therefore, the proponents of IT argue that the initial institutional and technical conditions as well as the macroeconomic environment are important but not critical for introducing IT, i.e., the lack of these conditions can be remedied after the introduction of this monetary regime \citep{Batini2007, IMF2006, Mishkin2007}. Given the lack of consensus in the empirical literature on the necessary preconditions for the implementation of IT, \citet{neumann2002} are probably right when concluding that the choice between IT and other monetary policy strategies is more a question of culture than economic considerations.

\section{Macroeconomic effects of IT}

Since the 2000s, there has been a growing interest in the effects of IT on macroeconomic performance, leading to diverse conclusions regarding the effectiveness of this monetary framework. This is particularly true for the empirical research focusing on EMEs, which represent a heterogeneous group with specific institutional and macroeconomic characteristics.

\subsection{The effects on inflation expectations and inflation persistence}

IT is a monetary regime in which price stability is the main objective of monetary policy, accompanied by explicit quantitative targets for the medium-term inflation rate. In addition, within this policy framework, the central bank is characterized by a high degree of transparency and accountability. It is believed that the announcement of explicit inflation targets is instrumental in anchoring inflation expectations of financial markets participants and private agents in general. In this respect, if credible, the IT regime will affect the formation of inflation expectations by reducing the backward-looking component and making them more forward-looking. In turn, well-anchored inflation expectations will lead to a fundamental change in inflation dynamics by reducing or even eliminating inflation inertia.

Unsurprisingly, much of the empirical literature has been concerned with testing the presumed beneficial effects of IT on inflation expectations and inflation persistence. In this regard, most of the studies are based on surveys of inflation expectations of businesses, households, or professional forecasters, which are only available for OECD countries. In addition, some studies work with data extracted from bond markets. As for the methodological apparatus, apart from the OLS regression framework, a number of papers employ a variety of time-series analysis models, ranging from structural break tests and various autoregression models to GARCH and fractional integration. In what follows, we review the relevant literature on this subject matter.

\citet{huh1996} and \citet{lane1998} are examples of two early studies which examine the behavior of inflation expectations following the adoption of IT in the UK. Both papers cover similar time periods, spanning from the mid-1970s to mid-1990s, employ similar econometric techniques – Vector Autoregression (VAR), and obtain the same finding – inflation expectations have declined following the adoption of IT. Within the VAR framework, \citet{schmidt-hebbel2002} provide evidence on the favorable effects of IT on inflation expectations in Brazil, Chile, and Mexico, \citet{demertzis2010} show that IT has contributed significantly to anchoring inflation expectations in seven advanced economies, while \citet{corbo2001} obtain the same results for a sample of 26 countries. \citet{neumann2002} conduct an event study of monetary policy for nine advanced economies by comparing the effects of the 1978 and 1998 oil shocks. They find that long-term interest rates, as a proxy for inflation expectations, increased in both IT and non-IT countries, but to a lesser amount in the former, concluding that inflation targeters achieved larger credibility gains. \citet{johnson2003}, too, show that the announcement of inflation targets has reduced inflation expectations in five advanced economies during 1984-1998. \citet{gillitzer2015} find that inflation expectations in Australia are better anchored in the latter phase of IT.

\citet{gurkaynak2007} take an alternative approach to measuring the effects of IT on long-term inflation expectations, based on daily bond yields in Canada, Chile, and the USA. Specifically, they compare the behavior of long-term nominal and indexed bond yields in response to important economic events and confirm the beneficial effects of IT. In a similar fashion, \citet{gurkaynak2010} construct the zero-coupon yield curve separately for the nominal interest rates and for the real interest rates (for inflation-indexed bonds) in the USA, the UK, and Sweden. Then, they calculate the implied forward interest rates and obtain the inflation compensation as a difference between nominal and real forward interest rates for a ten-year horizon. Finally, they regress inflation compensation on the surprise components in the announcements of important economic and financial data and conclude that, while US long-term inflation expectations are highly responsive to domestic economic news, the UK and Swedish ones generally respond neither to domestic nor to foreign news, implying that inflation expectations are firmly anchored in these two IT countries. \citet{depooter2014} follow the same methodology and find that inflation expectations are well-anchored in Brazil, Chile, and Mexico. Working with treasury yield data, \citet{suh2021} confirm the above findings for a mixed sample comprised of advanced countries and EMEs, providing some evidence that the anchoring effect might be even stronger in the latter.

Based on a sample of 15 countries, \citet{ehrmann2015} finds that inflation expectations in IT countries are better anchored compared with non-targeters. As a result, policy rates in IT countries need to react less to the changes in inflation so that IT countries are less likely to face the zero lower bound constraint. In addition, he studies the success of IT in anchoring inflation expectations under different circumstances: when inflation is normal, when it is persistently high, and when it is persistently low. He finds that when inflation is persistently low, the role of IT in anchoring inflation expectations get weaker, i.e., inflation expectations become more dependent on lagged inflation rate, the dispersion of individual inflation forecasts increases, and inflation expectations revise down when actual inflation is lower than expected, but they do not respond when actual inflation is higher than expected. Recently, \citet{ehrmann2021} investigates the performance of different types of inflation targets (targets set as ranges, points, and points surrounded by tolerance bands, respectively) and finds that no type of inflation target is superior in anchoring of inflation expectations.

However, the empirical literature is not unison about the favorable effects of IT on inflation expectations. Among the early papers, \citet{freeman1995} evaluate the effectiveness of IT in the early inflation targeters (the UK, Canada, New Zealand, and Sweden), and provide mixed evidence that inflation targets may have lowered inflation expectations in only two out of the four analyzed countries. Similarly, \citet{debelle1996} finds that IT has led to lower inflation expectations in New Zealand, but the evidence for Canada is weak. \citet{johnson2002} analyses 11 developed countries and shows that, following the adoption of IT, inflation expectations declined immediately in New Zealand and Sweden, the effect is smaller and slower to develop in the case of Canada and Australia, while it vanishes in the United-Kingdom. Moreover, he finds that neither the variability of expected inflation nor the average absolute forecast error fell after the announcement of inflation targets. \citet{levin2004} provide some evidence for the effectiveness of IT in anchoring long-run inflation expectations in industrialized countries, though it is not the case with short- and medium-run inflation expectations. Also, they employ the event-study approach to five EMEs and conclude that both short- and long-term inflation expectations in these countries did not change markedly after the introduction of IT. Working with the VAR methodology, \citet{davis2014} investigate inflation expectations for a sample of 36 developing countries and advanced economies. For the former, they find that inflation expectations respond less to shocks in inflation and oil prices during the post-IT period, thus, providing evidence that IT is beneficial in anchoring expectations. On the other hand, they obtain weak results for the latter sample. Similar results can be found in \citet{capistran2010}, who examine the effect of IT on the dispersion of inflation expectations from professional forecasters for a panel of 25 countries. They find that, after controlling for the global inflation trend, disinflation periods, country-specific effects as well the level and variance of inflation, IT leads to lower long-term inflation expectations. However, the full favorable effect is visible only after three years following the adoption of IT; also, the anchoring effect is present only in the developing countries, but not in advanced economies.

A number of papers cast doubts about the effectiveness of IT in anchoring inflation expectations. For instance, on the basis of simple descriptive data for OECD countries, \citet{almeida1998} argue that the announcement of inflation targets did not affect inflation expectations, which continued to be either the same as before or higher than the announced inflation targets. Exploiting survey data on long-term inflation expectations for 15 advanced countries, \citet{castelnuovo2003} find that inflation expectations are well anchored in almost all countries, notwithstanding whether they are inflation targeters or not, i.e., the announcements of explicit quantitative inflation targets do not matter for inflation expectations. This is indicated by both the low and generally decreasing volatility of expectations as well as the low and decreasing correlation between the revisions in short-term inflation expectations and the news about macroeconomic variables. \citet{mohanty2015}, too, find that inflation expectations in both IT and non-IT countries in Asia are well-anchored and provide similar evidence. \citet{mohanty2022} find that IT does not seem to anchor inflation expectations in India, as these are determined predominantly by supply shocks.

In the context of EMEs, the empirical evidence on the effects of IT on inflation expectations is also mixed. For instance, \citet{capistran2010} examine the dispersion of inflation expectations from professional forecasters and find that IT has led to lower long-term inflation expectations. However, the anchoring effect is visible only after three years following the adoption of IT; also, the anchoring effect is present only in the developing countries, but not in advanced economies. \citet{lin2008} find no significant difference in the dispersion of inflation expectations between IT and non-IT countries. Similarly, \citet{zhang2015} investigate the effects of IT on inflation expectations in a sample of 14 developing countries and find that IT does not significantly affect the anchoring of inflation expectations. \citet{mohanty2015} show that inflation expectations in IT countries are better anchored compared with non-targeters, but only during the period of relatively high inflation.

To summarize, the empirical evidence on the effects of IT on inflation expectations and inflation persistence is mixed and often conditional on the sample and the methodology employed. While several studies confirm the beneficial effects of IT in anchoring inflation expectations, others find no significant difference between IT and non-IT countries.

\subsection{The effects on inflation rate and variability}

The primary objective of IT is to achieve and maintain a low and stable inflation rate. Consequently, the empirical literature has extensively investigated the effects of IT on inflation performance, especially on the average inflation rate and its variability. Most of the studies in this field compare the inflation performance of IT and non-IT countries during the post-IT period.

\citet{bernanke1997inflation} and \citet{mishkin2000} are among the first studies to investigate the effects of IT on inflation performance. Both studies compare the inflation performance of IT and non-IT countries and find that IT countries experienced a significant reduction in both the average inflation rate and its variability. \citet{corbo2001} conduct a similar analysis and confirm the findings of the previous studies. \citet{neumann2002} compare the inflation performance of IT and non-IT countries during the 1990s and find that IT countries achieved a lower and more stable inflation rate. \citet{vega2005} employ a propensity score matching methodology and find that IT has led to a significant reduction in the average inflation rate and its variability.

However, not all studies confirm the beneficial effects of IT on inflation performance. For instance, \citet{ball1999a} argue that IT does not produce superior inflation performance compared with other monetary policy regimes. \citet{walsh2009} find that the inflation performance of IT countries converged towards that of non-IT countries, implying that IT did not provide additional benefits in terms of inflation control. \citet{goncalves2009} and \citet{lin2010} find no significant difference in the inflation performance of IT and non-IT countries.

In the context of EMEs, the empirical evidence on the effects of IT on inflation performance is also mixed. For instance, \citet{mishkin2000} and \citet{mishkin2002} find that IT has led to a significant reduction in the average inflation rate and its variability in EMEs. \citet{fry-mckibbin2014} and \citet{thornton2017} find similar evidence for a sample of developing countries. However, \citet{lin2010} and \citet{lin2009} find no significant difference in the inflation performance of IT and non-IT countries in the context of EMEs.

To summarize, the empirical evidence on the effects of IT on inflation performance is mixed and often conditional on the sample and the methodology employed. While several studies confirm the beneficial effects of IT in achieving a lower and more stable inflation rate, others find no significant difference between IT and non-IT countries.

\subsection{The effects on interest rates and exchange rates}

Though most of the empirical literature deals with the macroeconomic effects of IT on inflation and output, few papers make a step forward and investigate the effects on other important policy variables, such as interest rates and exchange rates.

In principle, IT should be associated with lower nominal and real interest rates as well as with lower interest rate volatility provided that at least some of the following propositions are true: it has stabilizing effects on inflation expectations; it is effective in reducing average inflation rate and inflation volatility; it improves the inflation-output trade-off. However, the accumulated empirical evidence has not yet resulted in firm conclusions on this issue.

Among the early empirical papers, \citet{huh1996} and \citet{lane1998} analyze the UK experience within the VAR framework and find that both short- and long-term interest rates declined after the adoption of IT. Employing the same methodology, \citet{laubach1997}, \citet{mishkin1997}, and \citet{neumann2002} obtain the same results for a sample of several advanced economies. Similarly, \citet{almeida1998} estimate EGARCH models for six advanced economies and find that IT reduces short-term interest rate volatility.

On the other hand, several papers obtain opposite findings. For instance, \citet{freeman1995} evaluate the early experience with IT in UK, Canada, NZ, and Sweden. They find that IT has not have any effects on the time-series properties of nominal long-term interest rates, but it has led to higher long-term real interest rates.

The findings from the empirical studies based on mixed samples are equally contradictory, though some of them support the hypothesis of ineffectiveness of IT in advanced economies. For instance, employing the propensity score matching methodology, \citet{ardakani2018} find that the average treatment effect of IT on interest rate volatility is negative and statistically significant only in developing economies, but not in industrialized countries. Similarly, the estimates from the dynamic panel-data models in \citet{fratzscher2020} imply that IT leads to higher policy rates in the sub-sample of OECD economies, but not in non-OECD countries. Based on the difference-in-difference methodology on a sample of ten Asian economies, \citet{naqvi2009} find that IT does not affect short-term interest rate volatility. On the other hand, \citet{petursson2004} obtains opposite results for a sample of 29 inflation targeters, showing that IT reduces short-run nominal interest rates. Finally, \citet{imf2006} and \citet{batini2007} argue that, in EMEs, inflation targeters seem to have experienced lower volatility in real interest rates. However, their conclusions should be taken cautiously since they are derived from simple descriptive statistics.

The relationship between IT and exchange rates can be analyzed from two points of view: first, exchange rate regimes as a determinant in the decision process of whether to adopt IT; second, following the adoption of IT, its impact on the behavior of nominal and real exchange rates. The first issue has been covered in Section 2 so that in what follows we focus on the latter research topic.

A priori, it is difficult to gauge the impact of IT on exchange rates: on the one hand, the implementation of IT requires flexible exchange-rate regime, which might result in excessive volatility of real exchange rates mostly driven by the changes in relative prices of tradable goods; on the other hand, the built-in flexibility of the IT regime, including more flexible exchange rates, may provide the central bank with more effective tools for curbing macroeconomic volatility; finally, under the assumption that IT indeed leads to lower inflation volatility, it will inevitably translate into lower real exchange rate variability due to the stability of internal prices. For instance, within a small open-economy model, \citet{ball2008} show that credible IT regime is more similar to floating and managed floating regimes than to currency pegs or “fear of floating”. That is, their model predicts that IT countries have lower probability of exchange rate changes than pure floaters, but higher probability than currency peggers, suggesting that, in terms of exchange rate variability, IT countries stand somewhere between floaters and peggers.

In view of the above opposite arguments, this issue could be resolved only on empirical grounds. Among the early evidence, \citet{almeida1998} estimate GARCH models for seven OECD countries over 1986-1997 and find that exchange-rate volatility during the IT regime has increased as compared to the fixed exchange-rate period, but it is lower than the period of discretionary monetary policy. Studying the UK’s experience with IT based on a Bayesian VAR model, \citet{lane1998} find that IT has no effects whatsoever on exchange rates. Since exchange rate volatility may be more relevant for EMEs, the subsequent empirical research has focused on these countries, too. As a result, there are several studies that utilize mixed samples comprising both industrialized countries and EMEs. For instance, \citet{rose2007} studies whether inflation targeters suffer from higher exchange-rate volatility compared with non-targeters. Although the IT dummy is significantly negative in only 19 out of 66 variants of the regression model, nonetheless, he concludes that IT countries have lower exchange rate volatility than non-targeters.

\citet{fouejieu2013} provides similar evidence for 79 countries, confirming that, in terms of inflation and GDP growth rate, there were no differences between IT countries and non-targeters during the Crisis; however, he finds that inflation targeters experienced lower inflation volatility and lower interest rates. Comparing the performance of inflation targeters and non-targeters during the Crisis, \citet{roger2009, roger2010} obtain different results for low-income and high-income countries: in the former, both inflation targeters and non-targeters saw similar decline in growth rates, but inflation rose to a lesser extent in IT countries; in the latter, inflation targeters saw smaller decline in growth rates and slightly less increase in inflation. Similarly, \citet{fry-mckibbin2014} provide mixed results by showing that IT have insulated developed countries from the global downturn, but the results are inconclusive for emerging markets in which IT has not proved to be successful regime during the Crisis.

In contrast, \citet{barnebeck2015} find that, compared with other strategies, especially with fixed exchange rates, inflation targeters fared better during the Crisis in terms of output growth. The same conclusion is obtained by \citet{carvalho2011} for a sample of 51 advanced economies and EMEs. Specifically, he finds that inflation targeters have had better growth performance relative to non-targeters in the post-Crisis period; also, median inflation rate in inflation targeters was never lower than 1.5\%, while it went below zero in non-targeters, suggesting that the former countries were less likely to face deflation. In addition, \citet{kose2018} find that IT countries experienced higher real exchange rate volatility compared to non-targeters, while \citet{kocenda2018} show that inflation persistence in IT countries did not rise during the Crisis as it did in the non-targeters. Similarly, \citet{duong2021} finds that inflation targeters in EMEs had lower inflation than non-targeters during the Crisis. Finally, \citet{arsic2022} show that the favorable effects of IT on inflation, inflation volatility, and GDP volatility were especially pronounced in the aftermath of the Crisis, but it did not affect either inflation persistence or GDP growth.

\subsection{The effects on fiscal outcomes}

As mentioned above, the early literature regards fiscal discipline as one of the main prerequisites for adopting IT though the empirical support for this proposition is far from strong. However, this strand of empirical literature deals with the importance of fiscal variables on the likelihood of adopting IT, but not on the reverse relationship – how IT affects fiscal outcomes. In principle, the presumed favorable effects of IT on fiscal discipline may be rationalized as follows \citep{Minea_Tapsoba_2014}: first, fiscal authorities may have an incentive to improve fiscal discipline in order to support the central bank’s commitment to the inflation target; second, IT may improve fiscal performance by keeping inflation low, thus mitigating the erosion of the real value of tax revenues (the negative Olivera-Tanzi effect); third, the lower inflation volatility associated with IT should stabilize the tax base, which in turn would result in better tax collection. However, note that the latter two arguments are conditional on the favorable effects of IT on average inflation and inflation volatility.

This issue is especially relevant for EMEs, which have a long tradition of relying on seignorage as a source of revenue. In this regard, \citet{minea_2021} build a theoretical model in which government spending is financed by both taxes and seignorage, and the quality of institutions affects tax collection. They show that, under some circumstances related to the political costs of policy reforms, IT may induce the government to offset the decline in seignorage by initiating political reforms (e.g., by improving the quality of institutions, fighting corruption, etc.) in order to strengthen tax collection. Employing the propensity score matching method on a panel of 30 EMEs, \citet{lucotte_2012} finds that the adoption of IT has quantitatively important effects on public revenue. Using similar methodology on a sample of 53 developing countries, \citet{minea_2021} confirms that IT may have favorable effects on tax collection.

Although the available empirical evidence is quite limited, generally, it supports the proposition that IT improves fiscal discipline. For instance, \citet{miles_2007} investigates whether IT provides the same level of fiscal discipline as hard pegs. He finds that IT improves fiscal performance, but currency unions and currency boards (to a lesser extent) lead to tighter fiscal policy than inflation targets. Therefore, he concludes that IT does not provide better fiscal performance than exchange-rate pegs. Using the difference-in-difference methodology, \citet{abo-zaid_tuzemen_2012} find that IT improves the budget balance in developed countries only, while it does not matter in developing countries; in addition, they find that IT does not affect budget balance volatility in both groups of countries. \citet{combes_2014} test several hypotheses related to the combination of IT and fiscal rules as well as their sequencing. Working with a panel of 152 countries over 1990-2009, they show that, taken on its own, i.e., irrespective of the existence of fiscal rules, IT has quantitatively important effects on the budget balance. When accompanied by fiscal rules, IT leads to even larger improvement in the budget balance, but this joint favorable effect is present only in the countries that have adopted fiscal rules before the adoption of IT.

Several papers deal with this topic within the propensity matching methodology. \citet{minea_tapsoba_2014} provide evidence that IT is associated with higher cyclically-adjusted fiscal balance and lower debt in developing countries, but not in developed ones. \citet{fry-mckibbin_wang_2014} find that inflation targeters in both developed economies and EMEs have lower tax burden and lower debt ratios than non-targeters, thus supporting the favorable effects of IT on fiscal discipline. \citet{ardakani_2018}, too, find that IT improves fiscal discipline in both developing and advanced economies with larger benefits in the latter. In the former, the adoption of IT encourages the fiscal authorities to pursue sound fiscal policy in order to support the credibility of the announced inflation targets. In the latter, governments are able to reap the credibility gains from the adoption of IT, which enables them to reduce the level of public debt. As a result, the favorable effect of IT on fiscal discipline is much larger in the advanced economies&#8203;:citation[oaicite:0]{index=0}&#8203;.

\subsection{Methodological issues}

The adoption of inflation targeting (IT) has been extensively studied in the empirical literature. However, several methodological issues need to be considered when interpreting the results of these studies. 

Firstly, there is the problem of selection bias. Countries that adopt IT may be systematically different from those that do not, in ways that are related to their macroeconomic performance. For instance, it is often argued that countries with a history of high inflation are more likely to adopt IT as a commitment device to reduce inflation \citep{BernankeMishkin1997}. To address this issue, some studies use propensity score matching or other econometric techniques to control for selection bias \citep{MishkinSavastano2001}.

Secondly, the measurement of IT itself can be problematic. There is no single definition of what constitutes IT, and different studies use different criteria to classify countries as inflation targeters \citep{Roger2010}. Some studies use the official adoption date of IT, while others consider the implementation of IT-like policies before the official adoption. This can lead to differences in the estimated effects of IT \citep{BallSheridan2005}.

Thirdly, the empirical literature often faces the challenge of isolating the effects of IT from other concurrent economic reforms. Many countries that adopt IT also implement other macroeconomic and structural reforms, making it difficult to attribute changes in economic performance solely to IT \citep{BatiniLaxton2007}.

\subsection{Effects of IT on macroeconomic performance}

\subsubsection{Inflation and inflation variability}

The primary goal of IT is to achieve and maintain low and stable inflation. The empirical evidence on the effectiveness of IT in achieving this goal is mixed. Some studies find that IT significantly reduces both the level and variability of inflation \citep{MishkinSchmidtHebbel2007}, while others find no significant difference between inflation targeters and non-targeters \citep{BallSheridan2005}.

For example, \citet{bernankeetal1999} find that IT countries experienced significant reductions in inflation and inflation variability compared to non-targeters. Similarly, \citet{vegawinkelried2005} find that IT leads to lower inflation and less inflation variability in both advanced and emerging market economies. However, \citet{ballsheridan2005} argue that the apparent success of IT may be due to regression to the mean, as countries with high initial inflation rates tend to experience larger reductions in inflation regardless of their monetary policy regime.

\subsubsection{Output growth and output variability}

The impact of IT on output growth and output variability is also a subject of debate. Some studies find that IT has no significant effect on output growth, while others find that it leads to more stable output growth \citep{Roger2010}.

\citet{mishkinsavastano2001} argue that IT can help stabilize output by anchoring inflation expectations, which reduces uncertainty and promotes investment. However, \citet{ballsheridan2005} find no significant difference in output growth or output variability between IT and non-IT countries.

\subsubsection{Interest rates and exchange rates}

The adoption of IT can also affect interest rates and exchange rates. \citet{svensson2000} argues that IT can lead to lower and more stable interest rates by anchoring inflation expectations. Similarly, \citet{mishkin2000} finds that IT countries tend to have lower and more stable interest rates compared to non-targeters.

The impact of IT on exchange rates is more complex. While IT is generally associated with greater exchange rate flexibility, the empirical evidence on the impact of IT on exchange rate volatility is mixed. Some studies find that IT leads to greater exchange rate volatility, while others find no significant difference between IT and non-IT countries \citep{Ball2000}.

\subsection{Disinflation costs of IT}

The empirical literature on the disinflation costs of IT, often measured by the sacrifice ratio (the output loss associated with reducing inflation), provides mixed results. Some studies find that IT reduces the sacrifice ratio by anchoring inflation expectations and reducing inflation persistence \citep{Mishkin2000}. However, other studies find no significant difference in the sacrifice ratio between IT and non-IT countries \citep{Ball1994}.

\citet{mishkin2000} argues that IT can reduce the sacrifice ratio by making monetary policy more transparent and predictable, which helps anchor inflation expectations and reduces the output cost of disinflation. However, \citet{ball1994} finds that the sacrifice ratio is largely determined by the initial conditions and the credibility of the central bank, rather than the adoption of IT per se.

\subsection{Conclusion}

The empirical literature on the macroeconomic effects of IT provides mixed results. While some studies find that IT leads to lower and more stable inflation, others find no significant difference between IT and non-IT countries. Similarly, the evidence on the impact of IT on output growth, output variability, interest rates, and exchange rates is mixed. The disinflation costs of IT, measured by the sacrifice ratio, also show mixed results. Overall, the effectiveness of IT appears to depend on a variety of factors, including initial conditions, the credibility of the central bank, and the broader macroeconomic and institutional context.

\bibliographystyle{apalike}
\bibliography{Macroeconomic_Effects_of_Inflation_Targeting_A_Survey_of_the_Empirical_Literature}
\end{document}
